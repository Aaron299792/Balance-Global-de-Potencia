\documentclass[11pt]{article}

\usepackage[utf8]{inputenc}	% Para caracteres en español
\usepackage{amsmath,amsthm,amsfonts,amssymb,amscd}
\usepackage{multirow,booktabs}
\usepackage[table]{xcolor}
\usepackage{fullpage}
\usepackage{tikz}
\usetikzlibrary{arrows.meta,decorations.pathmorphing}
\usepackage{pgfplots}
\pgfplotsset{compat=1.17}
\usepgfplotslibrary{colormaps}
\usepackage{lastpage}
\usepackage{enumitem}
\usepackage{fancyhdr}
\usepackage{mathrsfs}
\usepackage{wrapfig}
\usepackage{tikz}
\usetikzlibrary{3d}
\usepackage{setspace}
\usepackage{hyperref}
\usepackage{calc}
\usepackage{multicol}
%\usepackage[spanish]{babel}
\usepackage{cancel}
\usepackage[retainorgcmds]{IEEEtrantools}
\usepackage{geometry}
\usepackage{amsmath}
\newlength{\tabcont}
\setlength{\parindent}{0.0in}
\setlength{\parskip}{0.05in}
\usepackage{empheq}
\usepackage{framed}
\usepackage{caption,subcaption}
\usepackage[most]{tcolorbox}
\usepackage{xcolor}
\colorlet{shadecolor}{orange!15}
\parindent 0in
\parskip 12pt
\geometry{margin=1in, headsep=0.25in}
\theoremstyle{definition}
\newtheorem{defn}{Definition}
\newtheorem{reg}{Rule}
\newtheorem{exer}{Exercise}
\newtheorem{note}{Note}
\newcommand{\identity}{1\kern-0.25em\text{l}}
\newcommand{\todoitem}[1]{\item[$\Box$] #1}    % Unchecked box
\newcommand{\doneitem}[1]{\item[$\CheckedBox$] #1}  % Checked box (from amssymb)

\pagenumbering{arabic}

\title{Anteproyecto}
\author{Aaron Sanabria Mart\'inez\footnote{aaron.sanabria@ucr.ac.cr} \\ Tutor: Ricardo Solano Piedra\footnote{risolano@itcr.ac.cr}}

\begin{document}
\maketitle
	%\thispagestyle{empty}
	\section{Objetivos}
		\subsection{Objetivo General}
		Implementar una Red Neuronal Informada por la Física (PINN) para la determinación de los perfiles radiales de densidad electrónica y temperatura electrónica en un plasma magnetizado cilíndrico que exhibe transporte clásico perpendicular al campo magnético.
		
		\subsection{Objetivos Específicos}
		\begin{enumerate}
			\item Formular las ecuaciones que relacionan la densidad electrónica y temperatura electrónica a partir de la teoría de transporte clásico perpendicular al campo magnético para la base del diseño de la PINN.
			\item Diseñar la arquitectura de la PINN de acuerdo al modelo teórico mediante la definición de las capas, las funciones de activación y los hiperparámetros esenciales de la red. 
			\item  Validar los resultados de la PINN con una comparación de los perfiles obtenidos experimentales del  [Dispositivo Nombre] para evaluar el grado de fidelidad y precisión del modelo computacional implementado. 
		\end{enumerate}
	
  %%%%%%%%%%%%%%%%%%%%%%%%%%%%%%%%%%%%%%%%%%%%%%%%%%%%%%%%%%%%%%%%%%%%%%%%%%%%%%%%%%%%%%%%%%%%%%%%%%%%%%
  %\bibliographystyle{IEEEtran}
  %\bibliography{references.bib}
\end{document}