\section{M\'etodo usado en el art\'iculo (Aproximaciones)}

  Se definen las siguientes expresiones correspondientes a procesos at\'omicos en el Plasma:

  \begin{itemize}
    \item $N_{ion}$: N\'umero de ionizaciones por unidad $s\cdot m^3$.
    \item $N_{ion_2}$: N\'umero de segundas ionizaciones por unidad $s\cdot m^3$.
    \item $N_{rec}$: N\'umero de recombinaciones por unidad $s\cdot m^3$.
    \item $N_{rad}$: N\'umero de exitaciones por unidad $s\cdot m^3$.
    \item $Q_{ei}$: Densidad de potencia tranferida a los iones.
    \item $E_{ion}$: Energ\'ia de ionizaci\'on.
    \item $E_{rec}$: Energ\'ia cin\'etica de un electr\'on.
    \item $E_{rad}$: Promedio sobre todas las energ\'ias de radiaci\'on.
  \end{itemize}

  Las fuentes y sumideros de potencia relacionados a diferentes procesos est\'an dados por $Q_p = N_pE_p$. Las cantidades f\'isicas se definen matem\'aticamente como:

  \begin{eqnarray}
    N_{ion} &=& \left<\sigma v\right>_{ion}n_en_0\\
    N_{ion_2} &=& \left<\sigma v\right>_{ion_2}n_en_i\\
    N_{rec} &=& \left<\sigma v\right>_{rec}n_en_i\\
    N_{rad}^{0/i} &=& \left<\sigma v\right>_{rad}^{0/i}n_en_{0/i}\\
    Q_{ei} &=& n_e\frac{\frac{3}{2}(T_e - T_i)}{\tau^{ei}} \label{eq:Qei} \\
    E_{rec} &=& \frac{3}{2}T_e
  \end{eqnarray}

  En este caso $n_x$ son las densidades de part\'iculas de las correspondientes especies que hay en el plasma $x \in {i,e,0}$. Adem\'as, $e$, $i$, $0$ corresponden a electrones, iones y part\'iculas neutras respectivamente. La cantidad $\tau_{ei}$ es el tiempo de relajaci\'on para iones cargados simplemente y est\'a aproximado por:

  \begin{equation}\label{eq:tau}
    \tau_{ei} = \left(\frac{4\pi\epsilon_0}{e^2}\right)^2\frac{3m_iT_e^{3/2}}{8\sqrt{2\pi m_e}n_i\ln{\Lambda}}
  \end{equation}

  El trasporte radial en el plasma en equilibrio est\'a dado por:

  \begin{itemize}
    \item $\Gamma_n$: Transporte radial debido a fuentes y sumideros de part\'iculas.
    \item $\Gamma_E$: Transporte radial debido a fuentes y sumideros de energ\'ia electr\'onica.
  \end{itemize}

  Del balance de potencias por part\'iculas se tiene que

  \begin{equation}\label{eq:7}
    \Gamma_n = V(N_{ion} - N_{rec})
  \end{equation}

  \textbf{Las part\'iculas creadas por ionizaci\'on son ya sea recombinadas o trasportadas a las paredes}. De la te\'oria cl\'asica de trasporte se puede estimar este flujo como:

  \begin{equation}\label{eq:Gamma_n}
    \Gamma_n = -AD\nabla n
  \end{equation}

  Donde $A$ el \'area superficial estimada y $D$ la difusividad de part\'iculas en el plasma. El flujo total de potencia se balancea con fuentes y sumideros y se obtiene que 
  \begin{equation}\label{eq:8}
    \Gamma_E = P_{rf} - V(E_{ion}N_{ion} + E_{ion_2}N_{ion_2} + E_{rec}N_{rec} + E_{rad}^{0/i}N_{rad}^{0/i} + Q_{ei})
  \end{equation}

  Nuevamente de la teor\'ia cl\'asica de transporte se tiene que el flujo de potencia por energ\'ia est\'a dado por

  \begin{equation}\label{eq:9}
    \Gamma_E = -An_e\chi\nabla T_e + \frac{3}{2}\alpha T_e\Gamma_n
  \end{equation}

  donde $\chi$ es el coeficiente de difusividad de calor. Adem\'as esto asume que \textbf{la temperatura en la frontera del plasma es una fracci\'on de la temperatura del volumen}, i.e., $T_e(a) = \alpha T_e$ con $\alpha \in [0,1]$.
  
  Vamos a asumir lo siguiente para el caso f\'isico a estudiar:

  \begin{itemize}
    \item $T_i \ll T_e$: Temperatura i\'onica despreciable.
    \item $n_i \approx n_e = n$: Quasi-neutralidad del plasma.
    \item $n_0 \gg n$: Plasma parcialmente ionizado (Plasma frio).
    \item $\nabla T/ T \approx \nabla n / n$: Escala de cambio en los gradientes de densidad y temperatura similares.
  \end{itemize}
  
  Tomando la \eqref{eq:7} y reemplazando en \eqref{eq:9} se obtiene 

  \begin{equation}
    \Gamma_E = -A\chi n \nabla T_e - \frac{3}{2}\alpha ADT_e\nabla n
  \end{equation}

  Del \'utimo supuesto $n \nabla T_e \approx T_e \nabla n$ usando esto

  \begin{eqnarray}\label{eq:11}
    \Gamma_E &=& \left(\frac{3}{2}\alpha + \frac{\chi}{D}\right)(-AD\nabla n)T_e \nonumber \\
             &=& \left(\frac{3}{2}\alpha + \gamma\right)\Gamma_n T_e
  \end{eqnarray}

  Donde se define el parametro $\gamma = \chi/D$. Ahora usando el pen\'ultimo supuesto y despreciando los t\'erminos $n/n_0 \ll 1$ obtendremos una ecuaci\'on $n = n(T_e)$ multiplicando la ecuaci\'on \eqref{eq:11} por un factor de $1/n_0 V$ y reemplazando los respectivos $N_p$

  \begin{eqnarray*}
    \frac{\Gamma_E}{n_0V} &=& \small{\frac{P_{rf}}{n_0V} - \left(E_{ion}\frac{\left<\sigma v\right>_{ion} nn_0 }{n_0 } +\cancel{E_{ion_2}\left<\sigma v\right>_{ion_2} n \frac{n}{n_0} } + E_{rad}^{0/i}\left<\sigma v\right>_{rad}^{0/i}n \frac{n_{0/i}}{n_0} + \cancel{\frac{3}{2}T_e\left<\sigma v\right>_{rec}n \frac{n}{n_0} } + \frac{Q_{ei}}{n_0} \right)} \nonumber\\
                          &=& \frac{P_{rf}}{n_0V} -n \left(E_{ion}\left<\sigma v\right>_{ion} + E_{rad}^0\left<\sigma v\right>_{rad}^0 + \frac{Q_{ei}}{n_0}\right) 
  \end{eqnarray*}

  Ahora como $\tau_{ei} \propto n_i^{-1}$ y $Q_{ei} \propto \tau_{ei}^{-1}$ se tiene que $Q_{ei} \propto n_i$ lo que implica que $Q_{ei}/n_0 \propto n_i/n_0 \rightarrow 0$
  
  Por tanto se obtiene
  
  \begin{equation}\label{eq:8dev}
    \frac{\Gamma_{E}}{n_0V} \approx \frac{P_{rf}}{n_0V} -n (E_{ion}\left<\sigma v\right>_{ion} + E_{rad}^0\left<\sigma v\right>_{rad}^0)
  \end{equation}

  De forma similar se procede con la ecuaci\'on \eqref{eq:11} y reemplazando $\Gamma_n$ por la ecuaci\'ion \eqref{eq:7}

  \begin{eqnarray}\label{eq:11dev}
    \frac{\Gamma_E}{n_0 V} &=& \frac{1}{n_0V} \left(\gamma + \frac{3}{2}\alpha\right)V(N_{ion} - N_{rec})T_e \nonumber\\
                           &=& \left(\gamma + \frac{3}{2}\alpha\right)nT_e \left(\left<\sigma v\right>_{ion}\frac{n_0}{n_0} - \cancel{\left<\sigma v\right>_{rec}\frac{n}{n_0}}\right) \nonumber \\
                           &=& \left(\gamma + \frac{3}{2}\alpha\right)\left<\sigma v\right>_{ion}nT_e
  \end{eqnarray}

  Ahora igualamos las ecuaciones \eqref{eq:8dev} y \eqref{eq:11dev} y despejamos $n$

  \begin{eqnarray*}
    \frac{P_{rf}}{n_0V} -n (E_{ion}\left<\sigma v\right>_{ion} + E_{rad}^0\left<\sigma v\right>_{rad}^0) = \left(\gamma + \frac{3}{2}\alpha\right)\left<\sigma v\right>_{ion}nT_e \\ \implies \frac{P_{rf}}{n_0V} = n \left[\left(\gamma + \frac{3}{2}\alpha\right)\left<\sigma v\right>_{ion}T_e + E_{ion}\left<\sigma v\right>_{ion} + E_{rad}^0\left<\sigma v\right>_{rad}^0\right] \\
    \implies \frac{ P_{rf} }{n_0V} = n \left\{ \left[ \left(\gamma + \frac{3}{2}\alpha\right)T_e + E_{ion}\right] \left<\sigma v\right>_{ion} + E_{rad}^0\left<\sigma v\right>_{rad}^0 \right\}
  \end{eqnarray*}

  Finalmente se despeja $n$

  \begin{equation}
    n(T_e; P_{rf}, n_0, V, \gamma, \alpha) = \frac{P_{rf}/n_0V}{\left[ \left(\gamma + \frac{3}{2}\alpha\right)T_e + E_{ion}\right] \left<\sigma v\right>_{ion}(T_e) + E_{rad}^0\left<\sigma v\right>_{rad}^0(T_e)}
  \end{equation}

  Esta expresi\'on es la que se obtiene por \cite{lechte2002}.

  \subsection{Forma diferencial lineal de la densidad $n = n(T_e)$}
  Nos interesa el caso menos aproximado en particular deshacernos de la aproximaci\'on de los gradientes. Ac\'a nos desharemos de la aproximaci\'on $\nabla n /n \approx \nabla T /T$. Nuevamente asumimos cuasi-neutralidad y un plasma fr\'io, i.e., $n = n_e \approx n_i$ y $n_0 \gg n$. Retomando nuevamete la ecuaci\'on \eqref{eq:8} e igualando a la ecuaci\'on \eqref{eq:9}, adem\'as de reemplazar los $N_p$

  \begin{eqnarray*}
    -A\left(n\chi\nabla T_e + \frac{3}{2}\alpha T_e D\nabla n\right) = \\ P_{rf} - V\left(E_{ion}\left<\sigma v\right>_{ion}nn_0 + E_{ion_2}\left<\sigma v\right>_{ion_2}n^2 + \frac{3}{2}T_e\left<\sigma v\right>_{rec}n^2 + E_{rad}^0\left<\sigma v\right>_{rad}^{0/i}n n_{0/i} + Q_{ei}\right) 
  \end{eqnarray*}

  Nuevamente dividimos ambos lados de la ecuaci\'on por $n_0 V$ y descartamos los t\'erminos $n/n_0$ como lo hicimos en la secci\'on anterior de all\'i obtenemos

  \begin{eqnarray*}
    -\frac{A}{V}\left(\frac{3}{2}\alpha T_e D\frac{\nabla n}{n_0}\right) = \\ \frac{P_{rf}}{n_0V} - \left(E_{ion}\left<\sigma v\right>_{ion} + E_{rad}^0\left<\sigma v\right>_{rad}^{0}\right)n \\
    \implies \frac{A}{V}\left(\frac{3}{2}\alpha T_e D\frac{\nabla n}{n_0}\right) - \left(E_{ion}\left<\sigma v\right>_{ion} + E_{rad}^0\left<\sigma v\right>_{rad}^{0  }\right)n = -\frac{P_{rf}}{n_0V}
    \end{eqnarray*}
 
    De all\'i obtenemos la ecuaci\'on diferencial parcial en $n$ de 

    \begin{equation}
      \nabla n - \frac{2}{3}\frac{n_0 V}{\alpha AD}\left(  E_{ion}\left<\sigma v\right>_{ion} + E_{rad}^0\left<\sigma v\right>_{rad}^0\right)\frac{n}{T_e} = - \frac{2}{3}\frac{P_{rf}}{\alpha AD}\frac{1}{T_e}
    \end{equation}

  Nos interesa la informaci\'on que podemos sacar de est\'a ecuaci\'on. Tambi\'en vale la pena considerar que esto tambi\'en obedece al caso en que $\nabla T_e$ no cambia de forma abrupta a lo largo del plasma ya que de ser as\'i habr\'ia que considerar el t\'ermino que contiene $\nabla T_e$. En resumen este nuevo modelo se da bajo los siguientes supuestos:

  \begin{itemize}
    \item Plasma cuasi-neutro: $n = n_e \approx n_i$.
    \item La temperatura en los bordes es una fracci\'on de la temperatura del volumen: $T_e(a) = \alpha T_e$ con $\alpha \in [0,1]$
    \item Plasma fr\'io: $n \ll n_0$.
    \item Temperatura i\'onica baja: $T_i \ll T_e$.
    \item La temperatura cambia de forma continua y suave a lo largo del plasma: $\nabla T_e \cdot \frac{n}{n_0} \ll 1$
  \end{itemize}

  Este modelo tambi\'en depende de cuatro par\'ametros los cuales son: $P_{rf}$, $V$, $n_0$ y $\alpha\varUpsilon$ donde $\varUpsilon \equiv A D$. Pese a que este modelo se da en t\'erminos de una expresi\'on diferencial sigue siendo un modelo relativamente f\'acil de manejar debido a la linealidad de la expresi\'on. 

  La expresi\'on final queda como 

  \begin{equation}
    \nabla n - \frac{2}{3}\frac{n_0 V}{\alpha \varUpsilon}\left(  E_{ion}\left<\sigma v\right>_{ion}(T_e) + E_{rad}^0\left<\sigma v\right>_{rad}^0(T_e)\right)\frac{n}{T_e} = - \frac{2}{3}\frac{P_{rf}}{\alpha \varUpsilon}\frac{1}{T_e}
  \end{equation}
  Podemos tomar la complejidad del problema un paso m\'as adelante lo que haremos en la siguiente secci\'on.

  
  \subsection{Densidad $n = n(T_e)$ en un plasma cuasi-neutral}

  En este caso mantendremos \'unicamente tres supuestos:

  \begin{itemize}
    \item Cuasi-neutralidad: $n = n_e \approx n_i$.
    \item Temperatura en los bordes: $T_e(a) = \alpha T_e$ con $\alpha \in [0,1]$.
    \item Temperatura i\'onica baja: $T_i \ll T_e$.
  \end{itemize}

  Igualando las expresiones \eqref{eq:8} y \eqref{eq:9}, usando \eqref{eq:Gamma_n}, dividiendo ambos lados de la ecuaci\'on por V y de las definiciones de los $N_p$ obtenemos:

  \begin{eqnarray*}
  \frac{A}{V}(n\chi\nabla T_e + \frac{3}{2}\alpha D T_e \nabla n) = \\
  -\frac{P_{rf}}{V} + (E_{ion}\left<\sigma v\right>_{ion} + E_{rad}^0\left<\sigma v\right>_{rad}^0)n_0 n + \\ \left(E_{ion_2}\left<\sigma v\right>_{ion_2} + \frac{3}{2}T_e\left<\sigma v\right>_{rec} + E_{rad}^i\left<\sigma v\right>_{rad}^i + \frac{Q_{ei}}{n^2}\right)n^2
\end{eqnarray*}

Ahora trabajamos un poco m\'as con la interacci\'on de Coulomb que hasta ahora no se hab\'ia considerado. Usando las expresiones de $Q_{ei}$ y de $\tau^{ei}$ se puede definir las constante $\varpi \equiv \left(\frac{e^2}{4\pi\epsilon_0}\right)^2\frac{4\sqrt{2\pi m_e}}{m_i}$ y se puede reescribir $Q_{ei}$ tal que primero escribimos $1 / \tau^{ei}$ usando la Ec. \eqref{eq:tau}

  \begin{equation*}
    \frac{1}{\tau^{ie}} = \left(\frac{e^2}{4\pi\epsilon_0}\right)^2\frac{8\sqrt{2\pi m_e}n}{3 m_i}\frac{\ln{\Lambda}}{T_e^{3/2}}
  \end{equation*} 

  Insertamos esto en la expresi\'on de $Q_{ei}$ Ec.\eqref{eq:Qei} tal que 

  \begin{eqnarray}
    Q_{ei} &=& \frac{3}{2}n(T_e - T_i) \left(\frac{e^2}{4\pi\epsilon_0}\right)^2\frac{8\sqrt{2\pi m_e}n}{3 m_i}\frac{\ln{\Lambda}}{T_e^{3/2}} \nonumber \\
           &=& \frac{3}{2}nT_e\left(1 - \cancel{\frac{T_i}{T_e}}\right) \left(\frac{e^2}{4\pi\epsilon_0}\right)^2\frac{8\sqrt{2\pi m_e}n}{3 m_i}\frac{\ln{\Lambda}}{T_e^{3/2}} \nonumber \\
           &=& \left[\left(\frac{e^2}{4\pi\epsilon_0}\right)^2\frac{4\sqrt{2\pi m_e}}{m_i}\right]\frac{n^2\ln{\Lambda}}{T_e^{1/2}} \nonumber \\
           &=& \varpi\frac{n^2\ln{\Lambda}}{T_e^{1/2}}
  \end{eqnarray}

  Al reemplazar $Q_{ei}$ se obtiene
  \begin{eqnarray*}
    \frac{3}{2}\frac{\alpha AD T_e}{V}\nabla n - \left(E_{ion_2}\left<\sigma v\right>_{ion_2} + \frac{3}{2}T_e\left<\sigma v\right>_{rec} + E_{rad}^i\left<\sigma v\right>_{rad}^i + \frac{\varpi \ln{\Lambda}}{\sqrt{T_e}}\right)n^2 + \\ \left(\chi\frac{A}{V}\nabla T_e - E_{ion}\left<\sigma v\right>_{ion}n_0 - E_{rad}^0\left<\sigma v\right>_{rad}^0n_0\right)n = - \frac{P_{rf}}{V}
  \end{eqnarray*}

  Finalmente podemos dejar $\nabla n$ sin factores y obtemos la expresi\'on final incluyendo todas las dependencias con $T_e$

  \begin{eqnarray}\label{eq:PDEnT}
    \nabla n - \frac{2}{3}\frac{V}{\alpha \varUpsilon T_e}\left(E_{ion_2}\left<\sigma v\right>_{ion_2}(T_e) + \frac{3}{2}T_e\left<\sigma v\right>_{rec}(T_e) + E_{rad}^i\left<\sigma v\right>_{rad}^i(T_e) + \frac{\varpi\ln{\Lambda}(T_e)}{\sqrt{T_e}}\right)n^2 + \nonumber\\ \frac{2}{3}\frac{1}{\alpha\varUpsilon T_e}\left[\chi\nabla T_e - n_0V\left(E_{ion}\left<\sigma v\right>_{ion}(T_e) - E_{rad}^0\left<\sigma v\right>_{rad}^0(T_e)\right)\right]n = -\frac{2}{3}\frac{P_{rf}}{\alpha \varUpsilon T_e}
  \end{eqnarray}

  Recordando que $\varpi$ es una constante, se obtiene una expresi\'on final que depende de cinco par\'ametros los cuales son: $P_{rf}$, $V$, $n_0$, $\alpha\varUpsilon$, y $\chi$. 

  N\'otese que la ecuaci\'on \eqref{eq:PDEnT} es una ecuaci\'on diferencial parcial no lineal, lo que hace que sea un problema con una complejidad bastante alta. A partirde aqu\'i trataremos de plantear un 'fit an\'alitico' para la densidad usando una red Neuronal F\'isica (PINN). 

