\subsection{Forma diferencial lineal de la densidad $n = n(T_e)$}
  Nos interesa el caso menos aproximado en particular deshacernos de la aproximaci\'on de los gradientes. Ac\'a nos desharemos de la aproximaci\'on $|\nabla n /n| \approx |\nabla T /T|$. Nuevamente asumimos cuasi-neutralidad y un plasma fr\'io, i.e., $n = n_e \approx n_i$ y $n_0 \gg n$. Retomando nuevamete la ecuaci\'on \eqref{eq:8} e igualando a la ecuaci\'on \eqref{eq:9}, adem\'as de reemplazar los $N_p$

  \begin{eqnarray*}
    -A\left(n\chi_\perp(r)\partial_r T(r) + \frac{3}{2}T(r) D_\perp\partial_r n(r)\right) = \\ P_{rf} - V\left(E_{ion}\left<\sigma v\right>_{ion}nn_0 + E_{ion_2}\left<\sigma v\right>_{ion_2}n^2 + \frac{3}{2}T(r)\left<\sigma v\right>_{rec}n^2 + E_{rad}^0\left<\sigma v\right>_{rad}^{0/i}n n_{0/i} + Q_{ei}\right) 
  \end{eqnarray*}

  Nuevamente dividimos ambos lados de la ecuaci\'on por $n_0 V$ y descartamos los t\'erminos $n/n_0$ como lo hicimos en la secci\'on anterior de all\'i obtenemos

  \begin{eqnarray*}
    -\frac{A}{V}\left(\frac{3}{2} T(r) D_\perp\frac{\partial_r n(r)}{n_0}\right) = \\ \frac{P_{rf}}{n_0V} - \left(E_{ion}\left<\sigma v\right>_{ion} + E_{rad}^0\left<\sigma v\right>_{rad}^{0}\right)n(r) \\
    \implies \frac{A}{V}\left(\frac{3}{2} T(r) D_\perp(r)\frac{\partial n(r)}{n_0}\right) - \left(E_{ion}\left<\sigma v\right>_{ion} + E_{rad}^0\left<\sigma v\right>_{rad}^{0  }\right)n(r) = -\frac{P_{rf}}{n_0V}
    \end{eqnarray*}
 
    De all\'i obtenemos la ecuaci\'on diferencial parcial en $n$ de 

    \begin{equation}
      \frac{dn(r)}{dr} - \frac{2}{3}\frac{n_0 V}{AD_\perp(r)}\left(  E_{ion}\left<\sigma v\right>_{ion} + E_{rad}^0\left<\sigma v\right>_{rad}^0\right)\frac{n(r)}{T(r)} = - \frac{2}{3}\frac{P_{rf}}{AD_\perp}\frac{1}{T(r)}
    \end{equation}

  Nos interesa la informaci\'on que podemos sacar de est\'a ecuaci\'on. Tambi\'en vale la pena considerar que esto tambi\'en obedece al caso en que $\partial_r T(r)$ no cambia de forma abrupta a lo largo del plasma ya que de ser as\'i habr\'ia que considerar el t\'ermino que contiene $\partial_rT(r)$. En resumen este nuevo modelo se da bajo los siguientes supuestos:

  \begin{itemize}
    \item Plasma cuasi-neutro: $n = n_e \approx n_i$.
    \item Plasma fr\'io: $n \ll n_0$.
    \item Temperatura i\'onica baja: $T_i \ll T_e$.
    \item La temperatura cambia de forma continua y suave a lo largo del radio del plasma: $\partial_rT(r) \cdot \frac{n}{n_0} \ll 1$
    \item Geometr\'ia cilindr\'ica del Plasma.
    \item Campos Uniformes y un regímen de transporte cl\'asico.
    \item La densidad de neutros es aproximadamente uniforme en el espacio.
  \end{itemize}

  Este modelo tambi\'en depende de cuatro par\'ametros los cuales son: $P_{rf}$, $V$, $n_0$. Además se define $\varUpsilon(r) \equiv A D_\perp(r)$. Pese a que este modelo se da en t\'erminos de una expresi\'on diferencial sigue siendo un modelo relativamente f\'acil de manejar debido a la linealidad de la expresi\'on. 

  La expresi\'on final queda como 

  \begin{equation}
    \frac{dn(r)}{dr} - \frac{2}{3}\frac{n_0 V}{\varUpsilon(r)}\left(  E_{ion}\left<\sigma v\right>_{ion}(T) + E_{rad}^0\left<\sigma v\right>_{rad}^0(T)\right)\frac{n(r)}{T(r)} = - \frac{2}{3}\frac{P_{rf}}{\varUpsilon(r)}\frac{1}{T(r)}
  \end{equation}
  Podemos tomar la complejidad del problema un paso m\'as adelante lo que haremos en la siguiente secci\'on.
