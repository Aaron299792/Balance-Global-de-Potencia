\subsection{Densidad $n = n(T_e)$ en un plasma cuasi-neutral}

  En este caso mantendremos \'unicamente tres supuestos:

  \begin{itemize}
    \item Cuasi-neutralidad: $n = n_e \approx n_i$.
    \item Temperatura en los bordes: $T_e(a) = \alpha T_e$ con $\alpha \in [0,1]$.
    \item Temperatura i\'onica baja: $T_i \ll T_e$.
  \end{itemize}

  Igualando las expresiones \eqref{eq:8} y \eqref{eq:9}, usando \eqref{eq:Gamma_n}, dividiendo ambos lados de la ecuaci\'on por V y de las definiciones de los $N_p$ obtenemos:

  \begin{eqnarray*}
  \frac{A}{V}(n\chi\nabla T_e + \frac{3}{2}\alpha D T_e \nabla n) = \\
  -\frac{P_{rf}}{V} + (E_{ion}\left<\sigma v\right>_{ion} + E_{rad}^0\left<\sigma v\right>_{rad}^0)n_0 n + \\ \left(E_{ion_2}\left<\sigma v\right>_{ion_2} + \frac{3}{2}T_e\left<\sigma v\right>_{rec} + E_{rad}^i\left<\sigma v\right>_{rad}^i + \frac{Q_{ei}}{n^2}\right)n^2
\end{eqnarray*}

Ahora trabajamos un poco m\'as con la interacci\'on de Coulomb que hasta ahora no se hab\'ia considerado. Usando las expresiones de $Q_{ei}$ y de $\tau^{ei}$ se puede definir las constante $\varpi \equiv \left(\frac{e^2}{4\pi\epsilon_0}\right)^2\frac{4\sqrt{2\pi m_e}}{m_i}$ y se puede reescribir $Q_{ei}$ tal que primero escribimos $1 / \tau^{ei}$ usando la Ec. \eqref{eq:tau}

  \begin{equation*}
    \frac{1}{\tau^{ie}} = \left(\frac{e^2}{4\pi\epsilon_0}\right)^2\frac{8\sqrt{2\pi m_e}n}{3 m_i}\frac{\ln{\Lambda}}{T_e^{3/2}}
  \end{equation*} 

  Insertamos esto en la expresi\'on de $Q_{ei}$ Ec.\eqref{eq:Qei} tal que 

  \begin{eqnarray}
    Q_{ei} &=& \frac{3}{2}n(T_e - T_i) \left(\frac{e^2}{4\pi\epsilon_0}\right)^2\frac{8\sqrt{2\pi m_e}n}{3 m_i}\frac{\ln{\Lambda}}{T_e^{3/2}} \nonumber \\
           &=& \frac{3}{2}nT_e\left(1 - \cancel{\frac{T_i}{T_e}}\right) \left(\frac{e^2}{4\pi\epsilon_0}\right)^2\frac{8\sqrt{2\pi m_e}n}{3 m_i}\frac{\ln{\Lambda}}{T_e^{3/2}} \nonumber \\
           &=& \left[\left(\frac{e^2}{4\pi\epsilon_0}\right)^2\frac{4\sqrt{2\pi m_e}}{m_i}\right]\frac{n^2\ln{\Lambda}}{T_e^{1/2}} \nonumber \\
           &=& \varpi\frac{n^2\ln{\Lambda}}{T_e^{1/2}}
  \end{eqnarray}

  Al reemplazar $Q_{ei}$ se obtiene
  \begin{eqnarray*}
    \frac{3}{2}\frac{\alpha AD T_e}{V}\nabla n - \left(E_{ion_2}\left<\sigma v\right>_{ion_2} + \frac{3}{2}T_e\left<\sigma v\right>_{rec} + E_{rad}^i\left<\sigma v\right>_{rad}^i + \frac{\varpi \ln{\Lambda}}{\sqrt{T_e}}\right)n^2 + \\ \left(\chi\frac{A}{V}\nabla T_e - E_{ion}\left<\sigma v\right>_{ion}n_0 - E_{rad}^0\left<\sigma v\right>_{rad}^0n_0\right)n = - \frac{P_{rf}}{V}
  \end{eqnarray*}

  Finalmente podemos dejar $\nabla n$ sin factores y obtemos la expresi\'on final incluyendo todas las dependencias con $T_e$

  \begin{eqnarray}\label{eq:PDEnT}
    \nabla n - \frac{2}{3}\frac{V}{\alpha \varUpsilon T_e}\left(E_{ion_2}\left<\sigma v\right>_{ion_2}(T_e) + \frac{3}{2}T_e\left<\sigma v\right>_{rec}(T_e) + E_{rad}^i\left<\sigma v\right>_{rad}^i(T_e) + \frac{\varpi\ln{\Lambda}(T_e)}{\sqrt{T_e}}\right)n^2 + \nonumber\\ \frac{2}{3}\frac{1}{\alpha\varUpsilon T_e}\left[\chi\nabla T_e - n_0V\left(E_{ion}\left<\sigma v\right>_{ion}(T_e) - E_{rad}^0\left<\sigma v\right>_{rad}^0(T_e)\right)\right]n = -\frac{2}{3}\frac{P_{rf}}{\alpha \varUpsilon T_e}
  \end{eqnarray}

  Recordando que $\varpi$ es una constante, se obtiene una expresi\'on final que depende de cinco par\'ametros los cuales son: $P_{rf}$, $V$, $n_0$, $\alpha\varUpsilon$, y $\chi$. 

  N\'otese que la ecuaci\'on \eqref{eq:PDEnT} es una ecuaci\'on diferencial parcial no lineal, lo que hace que sea un problema con una complejidad bastante alta. A partirde aqu\'i trataremos de plantear un 'fit an\'alitico' para la densidad usando una red Neuronal F\'isica (PINN). 
