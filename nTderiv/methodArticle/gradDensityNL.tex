\subsection{Densidad $n = n(r)$ en un plasma cuasi-neutral}

  En este caso mantendremos \'unicamente los supuestos:

  \begin{itemize}
    \item Cuasi-neutralidad: $n = n_e \approx n_i$.
    \item Temperatura i\'onica baja: $T_i \ll T_e$.
    \item Geometr\'ia cil\'indrica del plasma.
    \item Campo magn\'etico uniforme.
    \item Reg\'imen de transporte clásico.
  \end{itemize}

  Igualando las expresiones \eqref{eq:8} y \eqref{eq:9}, usando \eqref{eq:Gamma_n}, dividiendo ambos lados de la ecuaci\'on por V y de las definiciones de los $N_p$ obtenemos:

  \begin{eqnarray*}
  \frac{A}{V}\left(n(r)\chi_\perp(r)\partial_r T(r) + \frac{3}{2}D_\perp(r) T(r) \partial_r n(r)\right) = \\
  -\frac{P_{rf}}{V} + (E_{ion}\left<\sigma v\right>_{ion} + E_{rad}^0\left<\sigma v\right>_{rad}^0)n_0 n(r) + \\ \left(E_{ion_2}\left<\sigma v\right>_{ion_2} + \frac{3}{2}T\left<\sigma v\right>_{rec} + E_{rad}^i\left<\sigma v\right>_{rad}^i + \frac{Q_{ei}}{n(r)^2}\right)n(r)^2
\end{eqnarray*}

Ahora trabajamos un poco m\'as con la interacci\'on de Coulomb que hasta ahora no se hab\'ia considerado. Usando las expresiones de $Q_{ei}$ y de $\tau_{ei}$ se puede definir las constante $\varpi \equiv \left(\frac{e^2}{4\pi\epsilon_0}\right)^2\frac{4\sqrt{2\pi m_e}}{m_i}$ y se puede reescribir $Q_{ei}$ tal que primero escribimos $1 / \tau_{ei}$ usando la Ec. \eqref{eq:tau}

  \begin{equation*}
    \frac{1}{\tau_{ei}} = \left(\frac{e^2}{4\pi\epsilon_0}\right)^2\frac{8\sqrt{2\pi m_e}n}{3 m_i}\frac{\ln{\Lambda}}{T^{3/2}}
  \end{equation*} 

  Insertamos esto en la expresi\'on de $Q_{ei}$ Ec.\eqref{eq:Qei} tal que 

  \begin{eqnarray}
    Q_{ei} &=& \frac{3}{2}n(T(r) - T_i) \left(\frac{e^2}{4\pi\epsilon_0}\right)^2\frac{8\sqrt{2\pi m_e}n}{3 m_i}\frac{\ln{\Lambda}}{T(r)^{3/2}} \nonumber \\
           &=& \frac{3}{2}nT(r)\left(1 - \cancel{\frac{T_i}{T(r)}}\right) \left(\frac{e^2}{4\pi\epsilon_0}\right)^2\frac{8\sqrt{2\pi m_e}n(r)}{3 m_i}\frac{\ln{\Lambda}}{T(r)^{3/2}} \nonumber \\
           &=& \left[\left(\frac{e^2}{4\pi\epsilon_0}\right)^2\frac{4\sqrt{2\pi m_e}}{m_i}\right]\frac{n(r)^2\ln{\Lambda}}{T(r)^{1/2}} \nonumber \\
           &=& \varpi\frac{n(r)^2\ln{\Lambda}}{T(r)^{1/2}}\label{eq:Qie2}
  \end{eqnarray}

  Al reemplazar $Q_{ei}$ se obtiene
  \begin{eqnarray*}
    \frac{3}{2}\frac{AD_\perp(r) T(r)}{V}\partial_r n(r) - \left(E_{ion_2}\left<\sigma v\right>_{ion_2} + \frac{3}{2}T(r)\left<\sigma v\right>_{rec} + E_{rad}^i\left<\sigma v\right>_{rad}^i + \frac{\varpi \ln{\Lambda}}{\sqrt{T(r)}}\right)n(r)^2 + \\ \left(\chi_\perp(r)\frac{A}{V}\partial_r T(r) - E_{ion}\left<\sigma v\right>_{ion}n_0 - E_{rad}^0\left<\sigma v\right>_{rad}^0n_0\right)n(r) = - \frac{P_{rf}}{V}
  \end{eqnarray*}

  Finalmente podemos dejar $\partial_r n$ sin factores y obtemos la expresi\'on final incluyendo todas las dependencias con $T(r)$

  \begin{eqnarray}\label{eq:PDEnT}
    \frac{dn(r)}{dr} &-& \frac{2}{3}\frac{V}{\varUpsilon(r) T(r)}\left(E_{ion_2}\left<\sigma v\right>_{ion_2}[T(r)] + \frac{3}{2}T(r)\left<\sigma v\right>_{rec}[T(r)] + E_{rad}^i\left<\sigma v\right>_{rad}^i[T(r)] + \frac{\varpi\ln{\Lambda}[T(r)]}{\sqrt{T(r)}}\right)n(r)^2  \nonumber\\ &+& \frac{2}{3}\frac{1}{\varUpsilon T(r)}\left[\chi_\perp(r)\frac{dT(r)}{dr} - n_0V\left(E_{ion}\left<\sigma v\right>_{ion}[T(r)] - E_{rad}^0\left<\sigma v\right>_{rad}^0[T(r)]\right)\right]n(r) \nonumber\\ &=& -\frac{2}{3}\frac{P_{rf}}{\varUpsilon(r) T(r)}
  \end{eqnarray}

  Recordando que $\varpi$ es una constante, se obtiene una expresi\'on final que depende de tres par\'ametros los cuales son: $P_{rf}$, $V$, $n_0$. 

  N\'otese que la ecuaci\'on \eqref{eq:PDEnT} es una ecuaci\'on diferencial ordinaria no lineal, lo que hace que sea un problema con una complejidad bastante alta. A partirde aqu\'i trataremos de plantear un 'fit an\'alitico' para la densidad usando una red Neuronal F\'isica (PINN). 
