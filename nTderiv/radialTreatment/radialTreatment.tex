\subsection{Modelo 1D de la descripción del Plasma}

Partiendo de la Ecs. \eqref{eq:supavg}, \eqref{eq:partflux} y \eqref{eq:energyflux} para el flujo radial de part\'iculas, suponiendo que el plasma está ionizado débilmente y que además el transporte de partículas es dominado bajo el régimen colisional; es decir, la termodifusividad de las partículas $D_T\nabla T$ es tal que $|D\nabla n| \gg |D_T\nabla T|$ se tiene que

\begin{eqnarray}
  \Gamma_n &=& -A\left<\textbf{D}\cdot\nabla n\right>_{\partial V} \\
           &=& \oint \textbf{D}\cdot\nabla n\cdot d\textbf{S}\\
           &=& \oint [D_\perp(r)\partial_r n(r)]\hat{r}\cdot\hat{r}dS(\phi,z) \\
           &=& -AD_\perp(r)\partial_r n(r) 
\end{eqnarray}


En cuanto el flujo de energía, también consideraremos únicamente el un transporte colisional para mantener la válidez del modelo. Para el primer término del flujo radial de energía se tiene que 

\begin{eqnarray}
  -A\left<\pmb{\chi}\cdot\nabla T\right>_{\partial V} &=& - \oint_{\partial V} (\pmb{\chi}\cdot\nabla T)\cdot d\textbf{S} \\
           &=& -\oint_{\partial V} (\chi_\perp(r) \partial_r T(r))\hat{r}\cdot\hat{r}dS(\phi,z)\\
           &=& -[\chi_\perp(r)\partial_rT(r)]\int dS(\phi,z)\\
           &=& -A\chi_\perp(r)\partial_rT(r)
\end{eqnarray}

El segundo término de Ec. \eqref{eq:energyflux} se puede simplificar como

\begin{eqnarray}
  \frac{3}{2}A\left<T\pmb{\Gamma}\right>_{\partial V} &=& \frac{3}{2}\oint T(r)\pmb{\Gamma}\cdot d\textbf{S} \\
                                                      &=& \frac{3}{2}T(r)\oint \pmb{\Gamma}\cdot d\textbf{S}\\
                                                      &=& \frac{3}{2}T(r)\Gamma_n
\end{eqnarray}

Estas suposiciones son válidas en el caso donde las variaciones poloidales y axiales de $T$ y $n$ son despreciables respecto a las dimensiones de el cambio radial, i.e., para una cantidad física macroscópica $\Delta(r,\phi, z) \approx \Delta(r)$ se tiene que, $|\partial_r \Delta| \gg |\partial_z\Delta|$, $|r^{-1}\partial_\phi\Delta|$.
