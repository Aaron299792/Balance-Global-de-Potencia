\section{Modelos Globales: Consideraciones Generales}

Los modelos globales son una primera aproximaci\'on en el estudio de plasmas. Representan un m\'etodo para describir al plasma basado en las ecuaciones cin\'eticas y de fluidos. Una de las primeras consideraciones que se deben hacer es sobre la geometr\'ia del problema. Por ejemplo, propiedades como la raz\'on entre la superficie interna del confinador y el volumen del plasma son de gran importancia en para estos modelos globales. Aqu\'i se desprecian derivadas espaciales o al menos se promedian. Sin embargo, su utilidad ya hace en describir r\'apidamente par\'arametros espacialmente promediados como densidad y temperatura\cite{hurlbatt2017}. La geometr\'ia m\'as simple de estudiar es un plasma cil\'indrico planar homog\'eneo. Otras consideraciones importantes son los m\'etodos usados para inyectar potencia al plasma.

En este contexto tambi\'en se deben plantear consideraciones sobre las especies en el plasma, en la gran mayor\'ia de casos se consideran distribuciones de Maxwell o en casos m\'as generales, $f_\alpha$ se aproxima como un distribuci\'on de Maxwell con una perturbaci\'on que introduce anisotr\'opias pequeñas en el plasma \cite{alves2018}. 

Se puede considerar a los modelos globales como una rama o un derivado del modelo de fluidos del plasma \cite{hurlbatt2017}. En sistemas reales, no conservativos, las ecuaciones tambi\'en deben considerar fuentes y sumideros de part\'iculas como veremos eventualmente. En los casos estacionarios no hay dependencias temporales impl\'icitas en el sistema, este sistema se basar\'a entonces en el estudio de las fuentes y sumideros del plasma y las perdidas en los bordes del flujo de part\'iculas. La energ\'ia se gana por las especies con carga atrav\'es de campos el\'ectricos, tambi\'en se considera la transferencia de energ\'ia entre especies iguales y entre especies distintas mediante procesos de colisiones el\'asticas y/o inel\'asticas. Tambi\'en, en estos modelos no se considera transporte, las relaciones entre temperatura $T_\alpha$ y densidad $n_\alpha$ no dependen de los gradientes, est\'a es una de las l\'imitaciones m\'as importantes del modelo. El modelo depende fuertemente en el conocimiento de las tasas de los procesos que se dan en el plasma y que determinan los t\'erminos de fuentes y sumideros del mismo. Si no se conocen adecuadamente estos coeficientes el modelo no convergera a ning\'un resultado coherente. 

La aplicaci\'on m\'as grande de estos modelos se da en Plasma de Baja Temperatura (LTP). A estas temperaturas para distribuciones no uniformes de densidad y energ\'ia se pierde gran cantidad de informaci\'on en el sistema debido al promedio espacial que este modelo implica \cite{hurlbatt2017}.

Para algunos sistemas donde las implicaciones espaciales son impr\'acticas o imposibles usar un m\'etodo semi-an\'alitico global puede ser una alternativa, pero en general para problemas con geometr\'ias m\'as complejas puede ser de mayor \'utilidad considerar modelos de mayor dimensionalidad as\'i como modelos donde la difusi\'on de energ\'ia y el transporte son en general caracterizadores del sistema. 
