\subsection{Procesos de calentemiento en el Plasma}

  Hasta el momento no se han hecho muchas consideraciones sobre el Plasma con el que se esta trabajando. Ahora para el modelo en estudio imponemos la condici\'on de un confinamiento en el r\'egimen de transporte cl\'asico. Esto es, los flujos de part\'iculas y energ\'ia obdecen la ley de Fick's es decir los flujos de part\'iculas y energ\'ia pueden ser descritos mediante los gradientes de densidad $n_\alpha$ y temperatura $T_\alpha$ en casos donde el campo $\textbf{B}$ es uniforme.

Retomemos el estudio de la cantidad $\textbf{Q}_\alpha$ que estudiamos en la secci\'on anterior, y añadamos las siguientes definiciones

  \begin{eqnarray}
    \textbf{q}_\alpha &\equiv& \frac{1}{2}m_\alpha n_\alpha\left<w_\alpha^2\textbf{w}_\alpha\right>\label{eq:q} \\
  \textbf{P}_\alpha &\equiv&  m_\alpha n_\alpha\left<\textbf{w}_\alpha\textbf{w}_\alpha\right>\label{eq:tensorP} \\
    p_\alpha &\equiv& \frac{1}{3}Tr(\textbf{P}_\alpha) = \frac{1}{3}m_\alpha n_\alpha\left<w_\alpha^2\right> \label{eq:pressureanisotropic}
  \end{eqnarray}

  Ahora profundizamos m\'as en los flujos de calor y las diferentes formas en las que la energ\'ia se difunde en el plasma. Comencemos partiendo de lo siguiente

  \begin{eqnarray*}
    \left<w_\alpha^2\textbf{w}_\alpha\right> = \left<(\textbf{v} - \textbf{u}_\alpha)\cdot(\textbf{v} - \textbf{u}_\alpha)\left[\textbf{v} - \textbf{u}_\alpha\right]\right> \\
    = \left<(v^2 + u_\alpha^2 - 2\textbf{u}_\alpha\cdot\textbf{v})\left[\textbf{v}-\textbf{u}_\alpha\right]\right> \\
    = \left< v^2\textbf{v} + u_\alpha^2\textbf{v} - 2(\textbf{u}_\alpha\cdot\textbf{v})\textbf{v} - v^2\textbf{u}_\alpha - u_\alpha^2\textbf{u}_\alpha + 2(\textbf{u}_\alpha\cdot\textbf{v})\textbf{u}_\alpha\right>\\
    = \left<v^2\textbf{v}\right> + u_\alpha^2\left<\textbf{v}\right> - u_\alpha^2\textbf{u}_\alpha - \left<v^2\right>\textbf{u}_\alpha - 2\textbf{u}_\alpha\cdot\left<\textbf{vv}\right> + 2\textbf{u}_\alpha\cdot\left<\textbf{vu}_\alpha\right>\\
  = \left<v^2\textbf{v}\right> + \cancel{u_\alpha^2\textbf{u}_\alpha} - \cancel{u_\alpha^2\textbf{u}_\alpha} - \left<v^2\right>\textbf{u}_\alpha - 2\textbf{u}_\alpha\cdot\left<\textbf{vv}\right> + 2\textbf{u}_\alpha\cdot\left<\textbf{vu}_\alpha\right>\\
    = \left<v^2\textbf{v}\right> -\left<v^2\right>\textbf{u}_\alpha - 2\textbf{u}_\alpha\cdot\left<\textbf{vv}\right> + 2\textbf{u}_\alpha\cdot\left<\textbf{vu}_\alpha\right>
  \end{eqnarray*}

  Ahora multiplicamos ambos lados de la ecuaci\'on por un factor de $\frac{1}{2}m_\alpha n_\alpha$ lo que resulta, usando las definiciones anteriores 
  
  \begin{eqnarray*}
    \textbf{q}_\alpha = \textbf{Q}_\alpha - \frac{1}{2}m_\alpha n_\alpha\left<v^2\right>\textbf{u}_\alpha - m_\alpha n_\alpha\textbf{u}_\alpha\cdot[\left<\textbf{vv} - \textbf{vu}_\alpha\right>]
  \end{eqnarray*}


  \begin{shaded}
  
    Usaremos el siguiente resultado para simplificar m\'as la relaci\'on
    
    \begin{eqnarray*}
    \textbf{u}_\alpha\cdot\left<\textbf{w}_\alpha\textbf{w}_\alpha\right> = \textbf{u}_\alpha\cdot\left<(\textbf{v} - \textbf{u}_\alpha)(\textbf{v} - \textbf{u}_\alpha)\right>\\
  = \textbf{u}_\alpha\cdot(\left<\textbf{vv}\right> + \left<\textbf{u}_\alpha\textbf{u}_\alpha\right> - \left<\textbf{u}_\alpha\textbf{v}\right> - \left<\textbf{vu}_\alpha\right>) \\
      = \textbf{u}_\alpha\cdot\left<\textbf{vv}\right> + \cancel{u_\alpha^2\textbf{u}_\alpha} - \cancel{u_\alpha^2\textbf{u}_\alpha} - \textbf{u}_\alpha\cdot\left<\textbf{vu}_\alpha\right>\\
      = \textbf{u}_\alpha\cdot[\left<\textbf{vv}\right> - \left<\textbf{vu}_\alpha\right>]
    \end{eqnarray*}

    Tambi\'en usaremos el siguiente resultado

    \begin{eqnarray*}
      \left<w^2_\alpha\right> = \left<v^2 + u_\alpha^2 - 2\textbf{u}_\alpha\cdot\textbf{v}\right>\\
      = \left<v^2\right> + u_\alpha^2 - 2\textbf{u}_\alpha\cdot\left<\textbf{v}\right>\\
      = \left<v^2\right> - u_\alpha^2
    \end{eqnarray*}
  \end{shaded}

  De all\'i es f\'acil ver que 

  \begin{eqnarray}
    \textbf{q}_\alpha &=& \textbf{Q}_\alpha -\frac{1}{2}m_\alpha n_\alpha u_\alpha^2 - \frac{1}{2}m_\alpha n_\alpha\left<w_\alpha^2\right> - \textbf{u}_\alpha\cdot\textbf{P}_\alpha\nonumber\\
  \implies \textbf{Q}_\alpha &=& \textbf{q}_\alpha + \frac{3}{2}T_\alpha n_\alpha\textbf{u}_\alpha - \frac{1}{2}m_\alpha n_\alpha u_\alpha^2\textbf{u}_\alpha - \textbf{P}_\alpha\cdot\textbf{u}_\alpha\nonumber \\
                        &=& \textbf{q}_\alpha + \frac{3}{2}T_\alpha\pmb{\Gamma}_\alpha - \frac{1}{2}m_\alpha u_\alpha^2\pmb{\Gamma}_\alpha - \textbf{P}_\alpha\cdot\textbf{u}_\alpha
  \end{eqnarray}

  N\'otese que por construcci\'on del tensor $\textbf{P}_\alpha$ este es anisotr\'opico, sin embargo, es diagonal y las contracciones $\textbf{u}_\alpha\cdot\textbf{P}_\alpha$ y $\textbf{P}_\alpha\cdot\textbf{u}_\alpha$ son equivalentes. Una forma alternativa de ver esto se logra con el tensor $\pmb{\Pi}_\alpha = \textbf{P}_\alpha - p_\alpha\pmb{1}$ 

  \begin{equation}
    \textbf{Q}_\alpha = \textbf{q}_\alpha + \frac{3}{2}T_\alpha\pmb{\Gamma}_\alpha + p_\alpha\textbf{u}_\alpha + \pmb{\Pi}_\alpha\cdot\textbf{u}_\alpha + \frac{1}{2}m_\alpha u_\alpha^2\pmb{\Gamma}_\alpha
  \end{equation}

  Se tienen los siguientes calentamientos en el plasma\cite{helander2005}

  \begin{itemize}
    \item $\textbf{q}_\alpha$ : flujo de calentamiento conductivo.
    \item $\frac{3}{2}T_\alpha\pmb{\Gamma}_\alpha$ : flujo de calentamiento convectivo.
    \item $p_\alpha\textbf{u}_\alpha$ : Entalp\'ia, trabajo que hace la presi\'on sobre el plasma.
    \item $\pmb{\Pi}_\alpha\cdot\textbf{u}_\alpha$ : transporte viscoso de energ\'ia.
    \item $\frac{1}{2}m_\alpha u_\alpha^2\pmb{\Gamma}_\alpha$ : convecci\'on de energ\'ia cin\'etica.
  \end{itemize}

Ahora bien, para un transporte en el cual la velocidad del flujo $\textbf{u}_\alpha$ surge en respuesta a gradientes suaves ($\delta << 1$) se puede asumir que est\'a cantidad es pequeña, $u \sim \delta v_T \ll v_T$ por lo que la podemos despreciar, a la vez, si las velocidad del flujo es negligible, tambi\'en podemos asumir que el transporte viscoso de energ\'ia es pequeño en comparaci\'on a la conducci\'on y convecci\'on. Finalmente, si asumimos que el cambio de volumen debido a la presi\'on es despreciable, podemos deshacernos del t\'ermino de entalp\'ia. Bajo estos supuestos\cite{dinklage2005}

  \begin{equation*}
    \textbf{Q}_\alpha \approx \textbf{q}_\alpha + \frac{3}{2}T_\alpha\pmb{\Gamma}_\alpha
  \end{equation*}

  De lo anterior el flujo de energ\'ia $\Gamma_E$ es dada por 

  \begin{equation}\label{eq:heatfluxp}
  \Gamma_E^\alpha \approx A\left<\textbf{q}_\alpha\right>_{\partial V} + A\frac{3}{2}\left<T_\alpha\pmb{\Gamma}_\alpha\right>_{\partial V}
  \end{equation}
