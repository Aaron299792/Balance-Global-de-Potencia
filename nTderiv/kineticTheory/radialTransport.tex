\subsection{Transporte radial y las leyes de Fick}

  De momento, bajo un transporte cl\'asico donde el campo magn\'etico es homogeneo asumimos que las leyes de Fick se sostienen tal que $q_j = -\chi\partial_j T$ y $\Gamma_j = -D\partial_j n$, m\'as adelante justificaremos esto. De aqu\'i se puede ver que finalmente las relaciones de transporte radial, o flujo de part\'iculas y energ\'ia se pueden escribir como

  \begin{eqnarray}
    \Gamma_n &=& -A\left<D\nabla n\right>_{\partial V}\\
    \Gamma_E &=& -A\left<\chi\nabla T\right>_{\partial V} - \frac{5}{2}A\left<T\pmb{\Gamma}\right>_{\partial V}
    \end{eqnarray}
    
  Las cantidades $\chi$ y $D$ deben ser estudiadas m\'as a fondo antes de asumir que son simplemente coeficientes, ya que podr\'an depender de la densidad $n$ y la temperatura $T$.
