\documentclass[11pt]{article}

\usepackage[utf8]{inputenc}	% Para caracteres en español
\usepackage{amsmath,amsthm,amsfonts,amssymb,amscd}
\usepackage{multirow,booktabs}
\usepackage[table]{xcolor}
\usepackage{fullpage}
\usepackage{lastpage}
\usepackage{enumitem}
\usepackage{fancyhdr}
\usepackage{mathrsfs}
\usepackage{wrapfig}
\usepackage{tikz}
\usetikzlibrary{3d}
\usepackage{setspace}
\usepackage{hyperref}
\usepackage{calc}
\usepackage{multicol}
\usepackage[spanish]{babel}
\usepackage{cancel}
\usepackage[retainorgcmds]{IEEEtrantools}
\usepackage{geometry}
\usepackage{amsmath}
\newlength{\tabcont}
\setlength{\parindent}{0.0in}
\setlength{\parskip}{0.05in}
\usepackage{empheq}
\usepackage{framed}
\usepackage[most]{tcolorbox}
\usepackage{xcolor}
\colorlet{shadecolor}{orange!15}
\parindent 0in
\parskip 12pt
\geometry{margin=1in, headsep=0.25in}
\theoremstyle{definition}
\newtheorem{defn}{Definition}
\newtheorem{reg}{Rule}
\newtheorem{exer}{Exercise}
\newtheorem{note}{Note}
\newcommand{\identity}{1\kern-0.25em\text{l}}
\newcommand{\todoitem}[1]{\item[$\Box$] #1}    % Unchecked box
\newcommand{\doneitem}[1]{\item[$\CheckedBox$] #1}  % Checked box (from amssymb)

\pagenumbering{arabic}

\title{Derivaci\'on de la densidad de iones y electrones $n = n(T_e)$ \\ \small{Notas sobre plasma basado en el art\'iculo de Lechte 2002} \cite{lechte2002}}
\author{Aaron Sanabria Mart\'inez\footnote{aaron.sanabria@ucr.ac.cr}}

\begin{document}
  \maketitle
	\thispagestyle{empty}

  \section{Colisiones en el Plasma: Secciones eficaces y frecuencias de colisi\'on}

  Las colisiones m\'as simples que pueden presentarse en un plasma son aquellas entre electrones con \'atomos neutros, estas pueden ser el\'asticas, lo que significa que ambas especies mantienen su identidad y en modelos simples los electrones simplemente rebotan de los neutros, se asume que el cambio de momento de los \'ultimos es negligible. Luego est\'an las colisiones inel\'asticas, en este caso se dan procesos de ionizaci\'on, radiaci\'on, recombinaci\'on. Entre especies con carga neta en el plasma tambi\'en se pueden encontrar las llamadas colisiones de Coulomb, que como su nombre lo indica se dan por interacciones de Coulomb entre las part\'iculas del plasma. 

  A groso modo, estos eventos mencionados pueden estudiarse desde la estad\'istica usando el concepto de probabilidad y secciones eficaces. Por ejemplo, para una colisi\'on el\'astica o inel\'astica se puede definir la secci\'on eficaz $\sigma$ que es el \'area transversal, si modelamos las part\'iculas como esferas, que tendr\'ia la part\'icula si al colisionar un electr\'on con el neutro, este se reflejara en un \'angulo de $90°$ o si el \'atomo se ionizara con todos los electrones que lo colisionan.

  Sea $n_0$ la densidad de neutros que hay en un plasma, se define el camino libre medio $\lambda_{mfp}$ como el recorrido promedio que tendr\'a el electr\'on antes de tener una probabilidad de colisionar con otro neutro \cite{goldston1995}. 

  \begin{equation}\label{eq:mfp}
    \lambda_{mfp} = (n_0\sigma)^{-1}
  \end{equation}

  Para electrones con velocidad $v$ el tiempo medio entre colisiones es

  \begin{equation}\label{eq:tmc}
    \tau = \lambda_{mfp}/v
  \end{equation}

  Al inverso de esta cantidad se le denomina "frecuencia de colisi\'on" y se define en t\'erminos de una funci\'on de distribuci\'on como
  
  \begin{equation}
    \nu = \left<\tau^{-1}\right> = n_0\left<\sigma v\right> = \frac{n_0}{n_e}\int d^3 f_e(v)\sigma(v)v
  \end{equation}

  Para cada tipo colisi\'on se puede asignar una frecuencia de colisi\'on de la cual se puede obtener la tasa en la que cada evento da paso a perdida o adici\'on de part\'iculas de una u otra especie en el plasma, al multiplicar estas frecuencias por la densidad de las part\'iculas se obtiene un t\'ermino $N_{evento}$ que es la tasa de emisi\'on o fuente, este es el n\'umero de eventos que se dan por segundo en un determinado volumen \cite{lechte2002}.

  \begin{equation}
    N_{evento}(T_\alpha) = n_\alpha n_\beta\left<\sigma v\right>_{evento}(T_\alpha) 
  \end{equation}

  En las colisiones in\'elasticas se pueden dar los siguientes eventos

  \begin{itemize}
    \item Ionizaci\'on por impacto: Al colisionar electrones con neutros u iones el proceso resulta en la liberaci\'on de dos electrones y un i\'on de menor $Z_\alpha$ que el original. 
    \item Recombinaci\'on de tres cuerpos: Dos electrones y un i\'on interactu\'an dando como resultado un neutro o un i\'on de mayor $Z_\alpha$ que el original.
    \item Ionizaci\'on radiativa: un fot\'on con suficiente energ\'ia interactu\'a con un neutro y como resultado se libera un electr\'on. 
  \item Recombinaci\'on radiativa: un electr\'on interact\'ua con un i\'on y se obtiene un neutro o un i\'on de mayor $Z_\alpha$ adem\'as de la emisi\'on de un fot\'on.
  \item Exitaci\'on: Debido a colisi\'on se transfiere energ\'ia al neutro o i\'on en ese proceso se irradia la energ\'ia de exitaci\'on mediante ondas electromagn\'eticas. En este caso siempre se pierde energ\'ia solamente.
  \item Colisiones de Coulomb: Son el\'asticas, los electrones calientes le transfieren momento a iones frios mediante interacciones el\'asticas de Coulomb.
  \end{itemize}  

  \subsection{Conceptos fundamentales de la teor\'ia cin\'etica y transporte en el Plasma}

  Los gases ionizados pueden ser espec\'ificados por una funci\'on de distribuci\'on $f_\alpha$(\textbf{r}, \textbf{v}, t). Una derivaci\'on Eur\'istica lleva a la conocida ecuaci\'on cin\'etica\cite{freidberg2014} 

  \begin{equation}\label{eq:k}
    \frac{df_\alpha}{dt} \equiv \left(\partial_t + \textbf{v}\cdot\nabla + \frac{Z_\alpha e}{m_\alpha}(\textbf{E} + \textbf{v}\times\textbf{B})\cdot \nabla_\nu \right)f_\alpha = C_\alpha
  \end{equation}

  \textbf{Esta ecuaci\'on no toma en cuenta fluctuaciones termicas}. Representa densidades suaves promediadas sobre un volumen con un gran n\'umero de part\'iculas. Los campos $\textbf{E}$ y $\textbf{B}$ tambi\'en varian suavemente, no hay fluctuaciones r\'apidas de microcampos y microfuerzas. Dichos efectos se consideran solo en el t\'ermino de colisiones $C_\alpha$.

  Ahora de la mec\'anica est\'istica es posible obtener los momentos de una distribuci\'on, nos interesa el primer y segundo momento\cite{helander2005} de la distribuci\'on definidos como 

  \begin{eqnarray}
    n_\alpha &=& \int d^3v f_\alpha \label{eq:dens}\\
    \textbf{u}_{\alpha} &=& \frac{1}{n_\alpha}\int d^3v f_\alpha\textbf{v} \label{eq:vel} \\
    \textbf{Q}_\alpha &\equiv& \frac{m_\alpha n_\alpha \left<v^2\textbf{v}\right>}{2}
  \end{eqnarray}

El momento cero en la Ec. \eqref{eq:dens} es la densidad de part\'iculas de una especie en el plasma, i.e., el n\'umero de part\'iculas de una especie por unidad de volumen. El primer momento Ec.\eqref{eq:vel} es la velocidad promedio de las part\'iculas de una especie en el plasma. La Ec.\eqref{eq:heat} es el flujo de energ\'ia. 

  Ahora definimos el flujo de part\'iculas en el plasma como 
  
  \begin{equation}
    \pmb{\Gamma}_\alpha \equiv n_\alpha\textbf{u}_\alpha
  \end{equation}

  De la ecuaci\'on cin\'etica \eqref{eq:k}, es posible obtener la siguiente ecuaci\'on de continuidad

  \begin{equation}\label{eq:cont}
    \partial_t n_\alpha + \nabla\cdot\pmb{\Gamma}_\alpha = S_\alpha
  \end{equation}

  donde $S_\alpha$ son las fuentes y/o sumideros de las part\'iculas $\alpha$. En estado estacionario como sugiere \cite{lechte2002} la Eq. \eqref{eq:cont} se reduce a 

  \begin{eqnarray*}
    \nabla\cdot\pmb{\Gamma}_\alpha = S_\alpha \\
    \implies \int_V d^3r\nabla\cdot\pmb{\Gamma}_\alpha = \int_V d^3r S_\alpha\\
    \implies \oint_{\partial V} \pmb{\Gamma}_\alpha\cdot d\textbf{S} = \int_V d^3r S_\alpha
  \end{eqnarray*}

  Definimos el flujo radial de part\'iculas como

  \begin{equation}\label{eq:rtpf}
    \Gamma_n^\alpha \equiv \oint_{\partial V} \pmb{\Gamma}_\alpha \cdot d\textbf{S} = \int_V d^3r S \approx \left<S\right>V
  \end{equation}

  Como es visible al adentrarse en balances globales, el concepto de promedios superficiales es fundamental. Conocer todos los parametros del plasma en su superficie ser\'a la piedra angular de los modelos que estamos por desarrollar.

  \begin{shaded}
    \textbf{Promedio Superficial}

    Definimos el promedio de una cantidad vectorial en la superficie como 

    \begin{equation}
      \left<\pmb{\varDelta}\right>_{\partial V} \equiv \frac{1}{A}\oint_{\partial V}\pmb{\varDelta}\cdot d\textbf{S}
    \end{equation}

    donde $A$ es el \'area superficial del plasma.
  \end{shaded}

  de lo anterior, podemos ver que el flujo radial de part\'iculas es simplemente el promedio superficial del flujo de part\'iculas que atraviesan la superficie, i.e., $\Gamma_n^\alpha = A\left<\Gamma_\alpha\right>_{\partial V}$ \cite{dinklage2005}.
  
  Ahora a partir de la llamada ecuaci\'on de energ\'ia \cite{helander2005} derivada de la Eq. \eqref{eq:k} se tiene que

  \begin{equation}\label{eq:energy}
    \frac{\partial}{\partial t}\left(\frac{3n_\alpha T_\alpha}{2} + \frac{m_\alpha n_\alpha u_\alpha^2}{2}\right) + \nabla\cdot\textbf{Q} = Z_\alpha e n_\alpha \textbf{E}\cdot\textbf{u}_\alpha + \int d^3v\frac{mv^2}{2}C_\alpha
  \end{equation} 

  En el lado izquierda de la Ec.\eqref{eq:energy} el primer t\'ermino corresponde al calentamiento de Joule en el plasma, para explicar el segundo se debe introducir los siguientes t\'erminos

  \begin{eqnarray}
    \textbf{w}_\alpha &\equiv& \textbf{v} - \textbf{u}_\alpha \nonumber\\
    \textbf{R}_\alpha &\equiv& \int d^3v m_\alpha\textbf{v}C_\alpha \nonumber\\
    \mathcal{Q}_\alpha &\equiv& \int d^3v \frac{mw_\alpha^2}{2}C_\alpha \nonumber\\
    \frac{3}{2}T_\alpha &\equiv& \left<\frac{m_\alpha w^2}{2}\right> \label{eq:temp}
  \end{eqnarray}

  Cada una de ellas corresponde a:
  \begin{itemize}
    \item $\textbf{w}_\alpha$ : desviaci\'on de la velocidad de la part\'icula de la velocidad promedio, i.e., velocidad por movimiento aleatorio. Cumple que $\left<\textbf{w}_\alpha\right> = \textbf{0}$.
    \item $\textbf{R}_\alpha$ : fuerza ejercida en la part\'icula como resultado de la colisi\'on con otras especies en el plasma.
    \item $\mathcal{Q}_\alpha$ : raz\'on de la energ\'ia t\'ermica transferida por colisiones con otras especies. 
  \end{itemize}

  El segundo t\'ermino en el lado izquierdo de la Ec.\eqref{eq:energy} es entonces, \textbf{la energ\'ia total transferida} por colisiones con otras especies.

  \begin{equation}
     \int d^3v \frac{m_\alpha v^2}{2}C_\alpha = \mathcal{Q}_\alpha + \textbf{R}_\alpha\cdot\textbf{u}_\alpha
  \end{equation}

  Aqu\'i el t\'ermino $\textbf{R}_\alpha\cdot\textbf{u}_\alpha$ corresponde al trabajo por unidad de tiempo realizado por la fuerza $\textbf{R}_\alpha$.

  B\'asicamente el lado izquierdo de la ecuaci\'on es la potencia debido al calentamiento de Joule o por colisiones con otras part\'iculas. En el caso estacionario la Ec. \eqref{eq:energy} se vuelve

  \begin{eqnarray}
  \nabla\cdot\textbf{Q} = \mathcal{Q}_\alpha + (Z_\alpha e n_\alpha\textbf{E} + \textbf{R}_\alpha)\cdot\textbf{u}_\alpha \nonumber\\
    \int_V d^3r \nabla\cdot\textbf{Q} = \mathcal{Q}_\alpha + (Z_\alpha e n_\alpha\textbf{E} + \textbf{R}_\alpha)\cdot\textbf{u}_\alpha\nonumber\\
    \oint_{\partial V} \textbf{Q}\cdot d\textbf{S} = \int_V d^3r\left[\mathcal{Q}_\alpha + (Z_\alpha e n_\alpha\textbf{E} + \textbf{R}_\alpha)\cdot\textbf{u}_\alpha\right] \nonumber
    \end{eqnarray}

    Finalmente llegamos a la siguiente expresi\'on al definir el flujo radial de energ\'ia $\Gamma_E \equiv \oint_{\partial V} \textbf{Q}\cdot d\textbf{S}$ y $Q \equiv \mathcal{Q}_\alpha + (Z_\alpha e n_\alpha\textbf{E} + \textbf{R}_\alpha)\cdot\textbf{u}_\alpha$.

    \begin{equation}
      \Gamma_E = \int_Vd^3r Q \approx \left<Q\right>V
    \end{equation}

    Trabajaremos con estas definiciones m\'as adelante.

  \subsection{Procesos de calentemiento en el Plasma}

  Hasta el momento no se han hecho muchas consideraciones sobre el Plasma con el que se esta trabajando. Ahora para el modelo en estudio imponemos la condici\'on de un confinamiento en el r\'egimen de transporte cl\'asico. Esto es, los flujos de part\'iculas y energía obdecen la ley de Fick's es decir los flujos de part\'iculas y energ\'ia pueden ser descritos mediante los gradientes de densidad $n_\alpha$ y temperatura $T_\alpha$ en casos donde el campo $\textbf{B}$ es uniforme.

  Por lo pronto dejaremos de lado los sub\'indices. Retomemos el estudio de la cantidad $\textbf{Q}$ que estudiamos en la secci\'on anterior, y añadamos las siguientes definiciones

  \begin{eqnarray}
    \textbf{q} &\equiv& \frac{1}{2}mn\left<w^2\textbf{w}\right>\label{eq:q} \\
    \textbf{P} &\equiv&  mn\left<\textbf{ww}\right>\label{eq:tensorP} \\
    p &\equiv& \frac{1}{3}Tr(\textbf{P}) = \frac{1}{3}mn\left<w^2\right> \label{eq:pressureanisotropic}
  \end{eqnarray}

  Ahora profundizamos m\'as en los flujos de calor y las diferentes formas en las que la energ\'ia se difunde en el plasma. Comencemos partiendo de lo siguiente

  \begin{eqnarray*}
    \left<w^2\textbf{w}\right> = \left<(\textbf{v} - \textbf{u})\cdot(\textbf{v} - \textbf{u})\left[\textbf{v} - \textbf{u}\right]\right> \\
    = \left<(v^2 + u^2 - 2\textbf{u}\cdot\textbf{v})\left[\textbf{v}-\textbf{u}\right]\right> \\
    = \left< v^2\textbf{v} + u^2\textbf{v} - 2(\textbf{u}\cdot\textbf{v})\textbf{v} - v^2\textbf{u} - u^2\textbf{u} + 2(\textbf{u}\cdot\textbf{v})\textbf{u}\right>\\
    = \left<v^2\textbf{v}\right> + u^2\left<\textbf{v}\right> - u^2\textbf{u} - \left<v^2\right>\textbf{u} - 2\textbf{u}\cdot\left<\textbf{vv}\right> + 2\textbf{u}\cdot\left<\textbf{vu}\right>\\
    = \left<v^2\textbf{v}\right> + \cancel{u^2\textbf{u}} - \cancel{u^2\textbf{u}} - \left<v^2\right>\textbf{u} - 2\textbf{u}\cdot\left<\textbf{vv}\right> + 2\textbf{u}\cdot\left<\textbf{vu}\right>\\
    = \left<v^2\textbf{v}\right> -\left<v^2\right>\textbf{u} - 2\textbf{u}\cdot\left<\textbf{vv}\right> + 2\textbf{u}\cdot\left<\textbf{vu}\right>
  \end{eqnarray*}

  Ahora multiplicamos ambos lados de la ecuaci\'on por un factor de $\frac{1}{2}mn$ lo que resulta, usando las definiciones anteriores 
  
  \begin{eqnarray*}
    \textbf{q} = \textbf{Q} - \frac{1}{2}mn\left<v^2\right>\textbf{u} - mn\textbf{u}\cdot[\left<\textbf{vv} - \textbf{vu}\right>]
  \end{eqnarray*}


  \begin{shaded}
  
    Usaremos el siguiente resultado para simplificar m\'as la relaci\'on
    
    \begin{eqnarray*}
      \textbf{u}\cdot\left<\textbf{ww}\right> = \textbf{u}\cdot\left<(\textbf{v} - \textbf{u})(\textbf{v} - \textbf{u})\right>\\
      = \textbf{u}\cdot(\left<\textbf{vv}\right> + \left<\textbf{uu}\right> - \left<\textbf{uv}\right> - \left<\textbf{vu}\right>) \\
      = \textbf{u}\cdot\left<\textbf{vv}\right> + \cancel{u^2\textbf{u}} - \cancel{u^2\textbf{u}} - \textbf{u}\cdot\left<\textbf{vu}\right>\\
      = \textbf{u}\cdot[\left<\textbf{vv}\right> - \left<\textbf{vu}\right>]
    \end{eqnarray*}

    Tambi\'en usaremos el siguiente resultado

    \begin{eqnarray*}
      \left<w^2\right> = \left<v^2 + u^2 - 2\textbf{u}\cdot\textbf{v}\right>\\
      = \left<v^2\right> + u^2 - 2\textbf{u}\cdot\left<\textbf{v}\right>\\
      = \left<v^2\right> - u^2
    \end{eqnarray*}
  \end{shaded}

  De all\'i es f\'acil ver que 

  \begin{eqnarray}
    \textbf{q} &=& \textbf{Q} -\frac{1}{2}mnu^2 - \frac{1}{2}mn\left<w^2\right> - \textbf{u}\cdot\textbf{P}\nonumber\\
    \implies \textbf{Q} &=& \textbf{q} + \frac{3}{2}Tn\textbf{u} - \frac{1}{2}mnu^2\textbf{u} - \textbf{P}\cdot\textbf{u}\nonumber \\
                        &=& \textbf{q} + \frac{3}{2}T\pmb{\Gamma} - \frac{1}{2}mnu^2\pmb{\Gamma} - \textbf{P}\cdot\textbf{u}
  \end{eqnarray}

  N\'otese que por construcci\'on del tensor $\textbf{P}$ este es anisotr\'opico, sin embargo, es diagonal y las contracciones $\textbf{u}\cdot\textbf{P}$ y $\textbf{P}\cdot\textbf{u}$ son equivalentes. Una forma alternativa de ver esto se logra con el tensor $\Pi_{ij} = P_{ij} - p\delta_{ij}$ y usando $p = nT$ 

  \begin{equation}
    Q_j = q_j + \frac{5}{2}pu_j + \Pi_{jk}u_k + \frac{1}{2}mu^2\Gamma_j
  \end{equation}

  Se tienen los siguientes calentamientos en el plasma\cite{helander2005}

  \begin{itemize}
    \item $\textbf{q}$ : flujo de calentamiento conductivo.
    \item $\frac{5}{2}p\textbf{u}$ : flujo de calentamiento convectivo.
    \item $\pmb{\Pi}\cdot\textbf{u}$ : transporte viscoso de energ\'ia.
    \item $\frac{1}{2}mu^2\pmb{\Gamma}$ : convecci\'on de energ\'ia cin\'etica.
  \end{itemize}

Ahora bien, para un transporte en el cual la velocidad del flujo $\textbf{u}$ surge en respuesta a gradientes suaves ($\delta << 1$) se puede asumir que est\'a cantidad es pequeña, $u \sim \delta v_T \ll v_T$ por lo que la podemos despreciar, a la vez, si las velocidad del flujo es negligible, tambi\'en podemos asumir que el transporte viscoso de energ\'ia es pequeño en comparaci\'on a la conducci\'on y convecci\'on. Bajo estos supuestos\cite{dinklage2005}

  \begin{equation*}
    Q_j \approx q_j + \frac{5}{2}T\Gamma_j
  \end{equation*}

  De lo anterior el flujo de energ\'ia $\Gamma_E$ es dada por 

  \begin{equation}\label{eq:heatfluxp}
    \Gamma_E \approx A\left<\textbf{q}\right>_{\partial V} + A\frac{5}{2}\left<T\pmb{\Gamma}\right>_{\partial V}
  \end{equation}

  \subsection{Transporte radial y las leyes de Fick}

  De momento, bajo un transporte cl\'asico donde el campo magn\'etico es homogeneo asumimos que las leyes de Fick se sostienen tal que $q_j = -\chi\partial_j T$ y $\Gamma_j = -D\partial_j n$, m\'as adelante justificaremos esto. De aqu\'i se puede ver que finalmente las relaciones de transporte radial, o flujo de part\'iculas y energ\'ia se pueden escribir como

  \begin{eqnarray}
    \Gamma_n &=& -A\left<D\nabla n\right>_{\partial V}\\
    \Gamma_E &=& -A\left<\chi\nabla T\right>_{\partial V} - \frac{5}{2}A\left<T\pmb{\Gamma}\right>_{\partial V}
    \end{eqnarray}
    
  Las cantidades $\chi$ y $D$ deben ser estudiadas m\'as a fondo antes de asumir que son simplemente coeficientes, ya que podr\'an depender de la densidad $n$ y la temperatura $T$.

  \section{M\'etodo usado en el art\'iculo (Aproximaciones)}

  Se definen las siguientes expresiones correspondientes a procesos at\'omicos en el Plasma:

  \begin{itemize}
    \item $N_{ion}$: N\'umero de ionizaciones por unidad $s\cdot m^3$.
    \item $N_{ion_2}$: N\'umero de segundas ionizaciones por unidad $s\cdot m^3$.
    \item $N_{rec}$: N\'umero de recombinaciones por unidad $s\cdot m^3$.
    \item $N_{rad}$: N\'umero de exitaciones por unidad $s\cdot m^3$.
    \item $Q_{ei}$: Densidad de potencia tranferida a los iones.
    \item $E_{ion}$: Energ\'ia de ionizaci\'on.
    \item $E_{rec}$: Energ\'ia cin\'etica de un electr\'on.
    \item $E_{rad}$: Promedio sobre todas las energ\'ias de radiaci\'on.
  \end{itemize}

  Las fuentes y sumideros de potencia relacionados a diferentes procesos est\'an dados por $Q_p = N_pE_p$. Las cantidades f\'isicas se definen matem\'aticamente como:

  \begin{eqnarray}
    N_{ion} &=& \left<\sigma v\right>_{ion}n_en_0\\
    N_{ion_2} &=& \left<\sigma v\right>_{ion_2}n_en_i\\
    N_{rec} &=& \left<\sigma v\right>_{rec}n_en_i\\
    N_{rad}^{0/i} &=& \left<\sigma v\right>_{rad}^{0/i}n_en_{0/i}\\
    Q_{ei} &=& n_e\frac{\frac{3}{2}(T_e - T_i)}{\tau^{ei}} \label{eq:Qei} \\
    E_{rec} &=& \frac{3}{2}T_e
  \end{eqnarray}

  En este caso $n_x$ son las densidades de part\'iculas de las correspondientes especies que hay en el plasma $x \in {i,e,0}$. Adem\'as, $e$, $i$, $0$ corresponden a electrones, iones y part\'iculas neutras respectivamente. La cantidad $\tau_{ei}$ es el tiempo de relajaci\'on para iones cargados simplemente y est\'a aproximado por:

  \begin{equation}\label{eq:tau}
    \tau_{ei} = \left(\frac{4\pi\epsilon_0}{e^2}\right)^2\frac{3m_iT_e^{3/2}}{8\sqrt{2\pi m_e}n_i\ln{\Lambda}}
  \end{equation}

  El trasporte radial en el plasma en equilibrio est\'a dado por:

  \begin{itemize}
    \item $\Gamma_n$: Transporte radial debido a fuentes y sumideros de part\'iculas.
    \item $\Gamma_E$: Transporte radial debido a fuentes y sumideros de energ\'ia electr\'onica.
  \end{itemize}

  Del balance de potencias por part\'iculas se tiene que

  \begin{equation}\label{eq:7}
    \Gamma_n = V(N_{ion} - N_{rec})
  \end{equation}

  \textbf{Las part\'iculas creadas por ionizaci\'on son ya sea recombinadas o trasportadas a las paredes}. De la te\'oria cl\'asica de trasporte se puede estimar este flujo como:

  \begin{equation}\label{eq:Gamma_n}
    \Gamma_n = -AD\nabla n
  \end{equation}

  Donde $A$ el \'area superficial estimada y $D$ la difusividad de part\'iculas en el plasma. El flujo total de potencia se balancea con fuentes y sumideros y se obtiene que 
  \begin{equation}\label{eq:8}
    \Gamma_E = P_{rf} - V(E_{ion}N_{ion} + E_{ion_2}N_{ion_2} + E_{rec}N_{rec} + E_{rad}^{0/i}N_{rad}^{0/i} + Q_{ei})
  \end{equation}

  Nuevamente de la teor\'ia cl\'asica de transporte se tiene que el flujo de potencia por energ\'ia est\'a dado por

  \begin{equation}\label{eq:9}
    \Gamma_E = -An_e\chi\nabla T_e + \frac{3}{2}\alpha T_e\Gamma_n
  \end{equation}

  donde $\chi$ es el coeficiente de difusividad de calor. Adem\'as esto asume que \textbf{la temperatura en la frontera del plasma es una fracci\'on de la temperatura del volumen}, i.e., $T_e(a) = \alpha T_e$ con $\alpha \in [0,1]$.
  
  Vamos a asumir lo siguiente para el caso f\'isico a estudiar:

  \begin{itemize}
    \item $T_i \ll T_e$: Temperatura i\'onica despreciable.
    \item $n_i \approx n_e = n$: Quasi-neutralidad del plasma.
    \item $n_0 \gg n$: Plasma parcialmente ionizado (Plasma frio).
    \item $\nabla T/ T \approx \nabla n / n$: Escala de cambio en los gradientes de densidad y temperatura similares.
  \end{itemize}
  
  Tomando la \eqref{eq:7} y reemplazando en \eqref{eq:9} se obtiene 

  \begin{equation}
    \Gamma_E = -A\chi n \nabla T_e - \frac{3}{2}\alpha ADT_e\nabla n
  \end{equation}

  Del \'utimo supuesto $n \nabla T_e \approx T_e \nabla n$ usando esto

  \begin{eqnarray}\label{eq:11}
    \Gamma_E &=& \left(\frac{3}{2}\alpha + \frac{\chi}{D}\right)(-AD\nabla n)T_e \nonumber \\
             &=& \left(\frac{3}{2}\alpha + \gamma\right)\Gamma_n T_e
  \end{eqnarray}

  Donde se define el parametro $\gamma = \chi/D$. Ahora usando el pen\'ultimo supuesto y despreciando los t\'erminos $n/n_0 \ll 1$ obtendremos una ecuaci\'on $n = n(T_e)$ multiplicando la ecuaci\'on \eqref{eq:11} por un factor de $1/n_0 V$ y reemplazando los respectivos $N_p$

  \begin{eqnarray*}
    \frac{\Gamma_E}{n_0V} &=& \frac{P_{rf}}{n_0V} - \left(E_{ion}\frac{\left<\sigma v\right>_{ion} nn_0 }{n_0 } +\cancel{E_{ion_2}\left<\sigma v\right>_{ion_2} n \frac{n}{n_0} } + E_{rad}^{0/i}\left<\sigma v\right>_{rad}^{0/i}n \frac{n_{0/i}}{n_0} + \cancel{\frac{3}{2}T_e\left<\sigma v\right>_{rec}n \frac{n}{n_0} } + \frac{Q_{ei}}{n_0} \right) \nonumber\\
                          &=& \frac{P_{rf}}{n_0V} -n \left(E_{ion}\left<\sigma v\right>_{ion} + E_{rad}^0\left<\sigma v\right>_{rad}^0 + \frac{Q_{ei}}{n_0}\right) 
  \end{eqnarray*}

  Ahora como $\tau_{ei} \propto n_i^{-1}$ y $Q_{ei} \propto \tau_{ei}^{-1}$ se tiene que $Q_{ei} \propto n_i$ lo que implica que $Q_{ei}/n_0 \propto n_i/n_0 \rightarrow 0$
  
  Por tanto se obtiene
  
  \begin{equation}\label{eq:8dev}
    \frac{\Gamma_{E}}{n_0V} \approx \frac{P_{rf}}{n_0V} -n (E_{ion}\left<\sigma v\right>_{ion} + E_{rad}^0\left<\sigma v\right>_{rad}^0)
  \end{equation}

  De forma similar se procede con la ecuaci\'on \eqref{eq:11} y reemplazando $\Gamma_n$ por la ecuaci\'ion \eqref{eq:7}

  \begin{eqnarray}\label{eq:11dev}
    \frac{\Gamma_E}{n_0 V} &=& \frac{1}{n_0V} \left(\gamma + \frac{3}{2}\alpha\right)V(N_{ion} - N_{rec})T_e \nonumber\\
                           &=& \left(\gamma + \frac{3}{2}\alpha\right)nT_e \left(\left<\sigma v\right>_{ion}\frac{n_0}{n_0} - \cancel{\left<\sigma v\right>_{rec}\frac{n}{n_0}}\right) \nonumber \\
                           &=& \left(\gamma + \frac{3}{2}\alpha\right)\left<\sigma v\right>_{ion}nT_e
  \end{eqnarray}

  Ahora igualamos las ecuaciones \eqref{eq:8dev} y \eqref{eq:11dev} y despejamos $n$

  \begin{eqnarray*}
    \frac{P_{rf}}{n_0V} -n (E_{ion}\left<\sigma v\right>_{ion} + E_{rad}^0\left<\sigma v\right>_{rad}^0) = \left(\gamma + \frac{3}{2}\alpha\right)\left<\sigma v\right>_{ion}nT_e \\ \implies \frac{P_{rf}}{n_0V} = n \left[\left(\gamma + \frac{3}{2}\alpha\right)\left<\sigma v\right>_{ion}T_e + E_{ion}\left<\sigma v\right>_{ion} + E_{rad}^0\left<\sigma v\right>_{rad}^0\right] \\
    \implies \frac{ P_{rf} }{n_0V} = n \left\{ \left[ \left(\gamma + \frac{3}{2}\alpha\right)T_e + E_{ion}\right] \left<\sigma v\right>_{ion} + E_{rad}^0\left<\sigma v\right>_{rad}^0 \right\}
  \end{eqnarray*}

  Finalmente se despeja $n$

  \begin{equation}
    n(T_e; P_{rf}, n_0, V, \gamma, \alpha) = \frac{P_{rf}/n_0V}{\left[ \left(\gamma + \frac{3}{2}\alpha\right)T_e + E_{ion}\right] \left<\sigma v\right>_{ion}(T_e) + E_{rad}^0\left<\sigma v\right>_{rad}^0(T_e)}
  \end{equation}

  Esta expresi\'on es la que se obtiene por \cite{lechte2002}.

  \section{Forma diferencial lineal de la densidad $n = n(T_e)$}
  Nos interesa el caso menos aproximado en particular deshacernos de la aproximaci\'on de los gradientes. Ac\'a nos desharemos de la aproximaci\'on $\nabla n /n \approx \nabla T /T$. Nuevamente asumimos cuasi-neutralidad y un plasma fr\'io, i.e., $n = n_e \approx n_i$ y $n_0 \gg n$. Retomando nuevamete la ecuaci\'on \eqref{eq:8} e igualando a la ecuaci\'on \eqref{eq:9}, adem\'as de reemplazar los $N_p$

  \begin{eqnarray*}
    -A\left(n\chi\nabla T_e + \frac{3}{2}\alpha T_e D\nabla n\right) = \\ P_{rf} - V\left(E_{ion}\left<\sigma v\right>_{ion}nn_0 + E_{ion_2}\left<\sigma v\right>_{ion_2}n^2 + \frac{3}{2}T_e\left<\sigma v\right>_{rec}n^2 + E_{rad}^0\left<\sigma v\right>_{rad}^{0/i}n n_{0/i} + Q_{ei}\right) 
  \end{eqnarray*}

  Nuevamente dividimos ambos lados de la ecuaci\'on por $n_0 V$ y descartamos los t\'erminos $n/n_0$ como lo hicimos en la secci\'on anterior de all\'i obtenemos

  \begin{eqnarray*}
    -\frac{A}{V}\left(\frac{3}{2}\alpha T_e D\frac{\nabla n}{n_0}\right) = \\ \frac{P_{rf}}{n_0V} - \left(E_{ion}\left<\sigma v\right>_{ion} + E_{rad}^0\left<\sigma v\right>_{rad}^{0}\right)n \\
    \implies \frac{A}{V}\left(\frac{3}{2}\alpha T_e D\frac{\nabla n}{n_0}\right) - \left(E_{ion}\left<\sigma v\right>_{ion} + E_{rad}^0\left<\sigma v\right>_{rad}^{0  }\right)n = -\frac{P_{rf}}{n_0V}
    \end{eqnarray*}
 
    De all\'i obtenemos la ecuaci\'on diferencial parcial en $n$ de 

    \begin{equation}
      \nabla n - \frac{2}{3}\frac{n_0 V}{\alpha AD}\left(  E_{ion}\left<\sigma v\right>_{ion} + E_{rad}^0\left<\sigma v\right>_{rad}^0\right)\frac{n}{T_e} = - \frac{2}{3}\frac{P_{rf}}{\alpha AD}\frac{1}{T_e}
    \end{equation}

  Nos interesa la informaci\'on que podemos sacar de est\'a ecuaci\'on. Tambi\'en vale la pena considerar que esto tambi\'en obedece al caso en que $\nabla T_e$ no cambia de forma abrupta a lo largo del plasma ya que de ser as\'i habr\'ia que considerar el t\'ermino que contiene $\nabla T_e$. En resumen este nuevo modelo se da bajo los siguientes supuestos:

  \begin{itemize}
    \item Plasma cuasi-neutro: $n = n_e \approx n_i$.
    \item La temperatura en los bordes es una fracci\'on de la temperatura del volumen: $T_e(a) = \alpha T_e$ con $\alpha \in [0,1]$
    \item Plasma fr\'io: $n \ll n_0$.
    \item Temperatura i\'onica baja: $T_i \ll T_e$.
    \item La temperatura cambia de forma continua y suave a lo largo del plasma: $\nabla T_e \cdot \frac{n}{n_0} \ll 1$
  \end{itemize}

  Este modelo tambi\'en depende de cuatro par\'ametros los cuales son: $P_{rf}$, $V$, $n_0$ y $\alpha\varUpsilon$ donde $\varUpsilon \equiv A D$. Pese a que este modelo se da en t\'erminos de una expresi\'on diferencial sigue siendo un modelo relativamente f\'acil de manejar debido a la linealidad de la expresi\'on. 

  La expresi\'on final queda como 

  \begin{equation}
    \nabla n - \frac{2}{3}\frac{n_0 V}{\alpha \varUpsilon}\left(  E_{ion}\left<\sigma v\right>_{ion}(T_e) + E_{rad}^0\left<\sigma v\right>_{rad}^0(T_e)\right)\frac{n}{T_e} = - \frac{2}{3}\frac{P_{rf}}{\alpha \varUpsilon}\frac{1}{T_e}
  \end{equation}
  Podemos tomar la complejidad del problema un paso m\'as adelante lo que haremos en la siguiente secci\'on.

  \section{Densidad $n = n(T_e)$ en un plasma cuasi-neutral}

  En este caso mantendremos \'unicamente tres supuestos:

  \begin{itemize}
    \item Cuasi-neutralidad: $n = n_e \approx n_i$.
    \item Temperatura en los bordes: $T_e(a) = \alpha T_e$ con $\alpha \in [0,1]$.
    \item Temperatura i\'onica baja: $T_i \ll T_e$.
  \end{itemize}

  Igualando las expresiones \eqref{eq:8} y \eqref{eq:9}, usando \eqref{eq:Gamma_n}, dividiendo ambos lados de la ecuaci\'on por V y de las definiciones de los $N_p$ obtenemos:

  \begin{eqnarray*}
  \frac{A}{V}(n\chi\nabla T_e + \frac{3}{2}\alpha D T_e \nabla n) = \\
  -\frac{P_{rf}}{V} + (E_{ion}\left<\sigma v\right>_{ion} + E_{rad}^0\left<\sigma v\right>_{rad}^0)n_0 n + \\ \left(E_{ion_2}\left<\sigma v\right>_{ion_2} + \frac{3}{2}T_e\left<\sigma v\right>_{rec} + E_{rad}^i\left<\sigma v\right>_{rad}^i + \frac{Q_{ei}}{n^2}\right)n^2
\end{eqnarray*}

Ahora trabajamos un poco m\'as con la interacci\'on de Coulomb que hasta ahora no se hab\'ia considerado. Usando las expresiones de $Q_{ei}$ y de $\tau^{ei}$ se puede definir las constante $\varpi \equiv \left(\frac{e^2}{4\pi\epsilon_0}\right)^2\frac{4\sqrt{2\pi m_e}}{m_i}$ y se puede reescribir $Q_{ei}$ tal que primero escribimos $1 / \tau^{ei}$ usando la Ec. \eqref{eq:tau}

  \begin{equation*}
    \frac{1}{\tau^{ie}} = \left(\frac{e^2}{4\pi\epsilon_0}\right)^2\frac{8\sqrt{2\pi m_e}n}{3 m_i}\frac{\ln{\Lambda}}{T_e^{3/2}}
  \end{equation*} 

  Insertamos esto en la expresi\'on de $Q_{ei}$ Ec.\eqref{eq:Qei} tal que 

  \begin{eqnarray}
    Q_{ei} &=& \frac{3}{2}n(T_e - T_i) \left(\frac{e^2}{4\pi\epsilon_0}\right)^2\frac{8\sqrt{2\pi m_e}n}{3 m_i}\frac{\ln{\Lambda}}{T_e^{3/2}} \nonumber \\
           &=& \frac{3}{2}nT_e\left(1 - \cancel{\frac{T_i}{T_e}}\right) \left(\frac{e^2}{4\pi\epsilon_0}\right)^2\frac{8\sqrt{2\pi m_e}n}{3 m_i}\frac{\ln{\Lambda}}{T_e^{3/2}} \nonumber \\
           &=& \left[\left(\frac{e^2}{4\pi\epsilon_0}\right)^2\frac{4\sqrt{2\pi m_e}}{m_i}\right]\frac{n^2\ln{\Lambda}}{T_e^{1/2}} \nonumber \\
           &=& \varpi\frac{n^2\ln{\Lambda}}{T_e^{1/2}}
  \end{eqnarray}

  Al reemplazar $Q_{ei}$ se obtiene
  \begin{eqnarray*}
    \frac{3}{2}\frac{\alpha AD T_e}{V}\nabla n - \left(E_{ion_2}\left<\sigma v\right>_{ion_2} + \frac{3}{2}T_e\left<\sigma v\right>_{rec} + E_{rad}^i\left<\sigma v\right>_{rad}^i + \frac{\varpi \ln{\Lambda}}{\sqrt{T_e}}\right)n^2 + \\ \left(\chi\frac{A}{V}\nabla T_e - E_{ion}\left<\sigma v\right>_{ion}n_0 - E_{rad}^0\left<\sigma v\right>_{rad}^0n_0\right)n = - \frac{P_{rf}}{V}
  \end{eqnarray*}

  Finalmente podemos dejar $\nabla n$ sin factores y obtemos la expresi\'on final incluyendo todas las dependencias con $T_e$

  \begin{eqnarray}\label{eq:PDEnT}
    \nabla n - \frac{2}{3}\frac{V}{\alpha \varUpsilon T_e}\left(E_{ion_2}\left<\sigma v\right>_{ion_2}(T_e) + \frac{3}{2}T_e\left<\sigma v\right>_{rec}(T_e) + E_{rad}^i\left<\sigma v\right>_{rad}^i(T_e) + \frac{\varpi\ln{\Lambda}(T_e)}{\sqrt{T_e}}\right)n^2 + \nonumber\\ \frac{2}{3}\frac{1}{\alpha\varUpsilon T_e}\left[\chi\nabla T_e - n_0V\left(E_{ion}\left<\sigma v\right>_{ion}(T_e) - E_{rad}^0\left<\sigma v\right>_{rad}^0(T_e)\right)\right]n = -\frac{2}{3}\frac{P_{rf}}{\alpha \varUpsilon T_e}
  \end{eqnarray}

  Recordando que $\varpi$ es una constante, se obtiene una expresi\'on final que depende de cinco par\'ametros los cuales son: $P_{rf}$, $V$, $n_0$, $\alpha\varUpsilon$, y $\chi$. 

  N\'otese que la ecuaci\'on \eqref{eq:PDEnT} es una ecuaci\'on diferencial parcial no lineal, lo que hace que sea un problema con una complejidad bastante alta. A partirde aqu\'i trataremos de plantear un 'fit an\'alitico' para la densidad usando una red Neuronal F\'isica (PINN). 
  %%%%%%%%%%%%%%%%%%%%%%%%%%%%%%%%%%%%%%%%%%%%%%%%%%%%%%%%%%%%%%%%%%%%%%%%%%%%%%%%%%%%%%%%%%%%%%%%%%%%%%
  \bibliographystyle{IEEEtran}
  \bibliography{references.bib}
\end{document}
